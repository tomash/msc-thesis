\section{Wstęp}

Przedmiotem niniejszej pracy jest badanie rezonansu stochastycznego w matematycznych modelach układach neuronów biologicznych, trakowanych jako układy z pamięcią i połączonych w sieci przestrzennie rozciągłe.

Rezonans stochastyczny jest zjawiskiem polegającym na wzmocnieniu przez układ słabego, zewnętrznego sygnału periodycznego w obecności szumu w taki sposób, że sygnał wynikowy (zachowanie układu) także jest periodyczny i o częstotliwości takiej jak częstotliwość owego zewnętrznego sygnału. Rezonans Stochastyczny obserwuje się w m.in. elementach progowych oraz wzbudnych (excitable). Matematyczne modele neuronów biologicznych wraz z rezonansem stochastycznym i układami z pamięcią stanowią trójkę gałęzi matematyki i fizyki, które składają się na przedmiot niniejszej pracy.

Ze względu na złożoną postać matematyczną, rozpatrywanie tego rodzaju zagadnień metodami analitycznymi jest trudne (w skrajnych przypadkach niemożliwe). Z tego powodu podstawą niniejszej pracy jest autorskie oprogramowanie napisane do symulacji badanych układów.

Pierwszym etapem pracy było zbadanie modelu pojedynczego neuronu biologicznego (element wzbudny), w którym w pewnym zakresie parametrów obserwuje się zjawisko rezonansu stochastycznego. Następnie z takich neuronów budowano jednowymiarowe macierze (łańcuchy) elementów oddziałujących ze sobą poprzez przekazywanie potencjałów czynnościowych.

W zależności od sposobu budowy takich układów zaobserwowano różne zjawiska. Najważniejsze z nich to wzmacnianie (kontrola) rezonansu stochastycznego w układach bez opoźnień i odbierających sygnał periodyczny w tej samej fazie oraz uzyskanie takiego samego wzmocnienia w układach odbierających sygnał periodyczny w różnej fazie (wpływ rozciągłości przestrzennej), ale z opóźnieniami w transmisji potencjałów kompensującymi różnice faz.