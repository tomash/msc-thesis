\documentclass[12pt]{article}

  \usepackage[polish]{babel}
  \usepackage[utf8]{inputenc}
  \usepackage[T1]{fontenc} 
  \usepackage{hyperref}
  
  \usepackage[numbers,sort&compress]{natbib}
  \usepackage{hypernat}

\begin{document}
  \section{Wstęp}
  Rezonans stochastyczny jest zjawiskiem obserwowanym w elementach progowych oraz wzbudnych (excitable). Matematyczne modele neuronów biologicznych wraz z rezonansem stochastycznym i układami z pamięcią stanowią trójkę młodych gałęzi matematyki i fizyki, które składają się na przedmiot niniejszej pracy.
  
  Wszystkie te dziedziny oprócz relatywnie młodego wieku mają jeszcze jedną wspólną cechę: sprawiają duże trudności w rozwiązywaniu metodami analitycznymi, w wyniku czego szeroko się w nich stosuje symulacje numeryczne.
  
  W pracy wyszedłem od modelu pojedynczego neuronu biologicznego (element wzbudny), w którym przy pewnych zakresach parametrów kontrolnych obserwuje się zjawisko rezonansu stochastycznego. Bazując na tym modelu i poruszając się w tym zakresie parametrów kontrolnych, rozszerzyłem go o połączenia międzyneuronowe, następnie dodając opóźnienie w przekazywaniu sygnałów do połączonych neuronów. Operując parametrami kontrolnymi, badałem przebiegi czasowe oraz signal-to-noise ratio (SNR) dla wszystkich neuronów w poszczególnych scenariuszach. 
  
  
  \section{Podstawy Teoretyczne}
  
  \subsection{Rezonans Stochastyczny}
  
  Rezonans stochastyczny (SR) jest zjawiskiem występującym w układach sterowanych co najmniej dwoma sygnałami, z których jeden jest periodyczny, drugi natomiast jest szumem. O występowaniu rezonansu stochastycznego mówimy, kiedy nałożenie sygnału periodycznego i niezerowego szumu skutkuje wysoką periodycznością sygnału wyjściowego.
  
  Podstawową wielkością używaną przy badaniu rezonansu stochastyznego jest stosunek sygnał-szum (signal-to-noise ratio, zwany dalej SNR), określający skuteczność wzmocnienia składowej periodycznej w wyniku działania szumu.
  
  Istnieje wiele modeli układów z rezonansem stochastycznym, od podwójnej studni potencjału, przez układy progowe po układy równań różniczkowych.
  
  \subsection{Modele Neuronów Biologicznych}
  
  Fizyczne pobudzenie posiadającego układ nerwowy zwierzęcia powoduje zmienne w czasie prądy jonowe w membranach neuronów sensorycznych, które w konsekwencji wywołują potencjały czuciowe (?) ("strzały" lub "igły") w momencie depolaryzacji membrany.
  
  Modele neuronów biologicznych/sensorycznych (sensory neuron model, SNM), podobnie jak neuronów binarnych używanych w badaniu sieci neuronowych, wykazują bistabilność, tzn. istnienie dwóch stanów stabilnych w których układ może przebywać. Podstawową różnicą jest ciągły potencjał (napięcie) SNM, co oznacza traktowanie stanów stabilnych jako zakresów (zamiast pojedynczych wartości), pomiędzy którymi układ przechodzi poprzez stany niestabilne. 
  
  Naturalną konsekwencją ciągłego potencjału jest użycie równań różniczkowych (zamiast różnicowych w neuronach binarnych) w matematycznym opisie SNM. Oprócz potencjału w wielu SNM istnieje równanie różniczkowe na drugą zmienną układu, ujemnie wpływającą na pochodną potencjału, zwaną dalej relaksacja.
  
  Pierwszym SNM był model AAA-BBB, za który w roku XXXX twórcy zostali uhonorowani nagrodą Nobla. Jest to model skomplikowany, o dużej ilości równań i zmiennych, zaproponowany na podstawie badań nad neuronami sensorycznymi CRABFISH?
  
  Model Fitzhugh-Nagumo (FHN), będący podstawą tej pracy, modeluje potencjał neuronu jako układ dwóch równań różniczkowych z jedną zmienną losową. Spełnia zatem podstawowe kryterium układu nieliniowego (chaotycznego).
  
  \subsection{Pamięć i Opóźnienie}
  
  
  \section{Symulacje Numeryczne}
  
  \subsection{Oprogramowanie}
  
  
  
  \subsection{Pojedynczy Neuron}
  
  Przed przystąpieniem do rozwiązywania układów wieloneuronowych rozpocząłem eksperyment od zbadania pojedynczego neuronu, jak opisywany w pracy A. Longtin \cite{longtin}. Dzięki temu mogłem sprawdzić zarówno poprawność własnego oprogramowania, jak i model oraz zestaw parametrów opisane w wyżej wymienionej publikacji.
  
  Stochastyczne równania różniczkowe składające się na model FHN całkowałem według metody opisanej w pracy Mannella, Palleschi \cite{mannella}. Po rozwiązaniu przedstawionego tam równania, krok całkowania po czasie dt liczony był następująco:
  
  Przebieg czasowy potencjału pojedynczego neuronu, z zestawem parametrów jak w \cite{longtin}.
  
  
  Jest to przebieg zgodny z zaobserwowanym w pracy A. Longtin.
  
  \subsection{Układy Neuronów Bez Opóźnienia}
  
  Następnym krokiem pracy było połączenie neuronów (początkowo dwóch) w "sznur", z sygnałem przekazywanym w jedną stronę, tzn. sygnał z neuronu i jest odbierany przez neuron i-1.
  
  Odbierany sygnał jest dodawany do potencjału po przemnożeniu przez stałą wymiany c, wspólną dla wszystkich neuronów. Dodatkowo, aby zapobiec, odbierane są wyłącznie sygnały o potencjale przekraczającym próg odcięcia lh, aby zapobiec wzmocnieniu samowzbudzenia wyłącznie poprzez kumulację słabej składowej harmonicznej.
  
  \subsection{Układy Neuronów Z Opóźnieniem}
  
  
  \newpage
  
  \addcontentsline{toc}{section}{Literatura}
  \bibliographystyle{unsrtnat}
  \bibliography{bibliografia}
  
  
\end{document}
