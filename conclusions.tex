\section{Wnioski}

Układy badane w niniejszej pracy to łańcuchy neuronów FitzHugh-Nagumo z przesunięciami fazowymi (modelującymi wpływ rozciągłości przestrzennej układu) sygnału periodycznego i opóźnieniem czasowym w transmisji sygnału między połączonymi elementami. W szczególności badano łańcuchy bez opóźnień transmisji i przesunięć fazowych, łańcuchy bez opóźnień i z przesunięciami fazowymi oraz kompensacyjny efekt opóźnień w łańcuchach z przesunięciami fazowymi.

Opracowano oprogramowanie, w szczególności implementujące metodę rozwiązywania stochastycznych równań różniczkowych do modelowania ww. układów neuronów FitzHugh-Nagumo.

W zależności od cech badanego układu zaobserwowano różne zjawiska.
\begin{enumerate}
  \item Wzmocnienie SR w układach bez opóźnień i przesunięć fazowych, będące analogiem zjawiska zwanego Array-Enhanced Stochastic Resonance.
  \item Osłabieie SR wskutek przesunięć fazowych przy braku odpowiednio dobranych opóźnień.
  \item Zjawisko wzmocnienia SR po wprowadzeniu odpowiednio dobranych opóźnień do układu przestrzennie rozciągłego, dobranych tak, by skompensować różnice faz pomiędzy oddziałującymi elementami.
\end{enumerate}
  
Pomimo zupełnie różnego modelu, można w macierzach połączonych neuronów FitzHugh-Nagumo zaobserwować zjawiska podobne do zaobserwowanych w macierzach prostych, progowych elementów. Wymaga to operowania w wąskim zakresie parametrów, w którym neuron FHN zachowuje się podobnie do elementu progowego.

Warunkiem zaobserwowania Array-Enhanced Stochastic Resonance w macierzach połączonych neuronów FitzHugh-Nagumo jest wprowadzenie opóźnień w transmisji sygnałów kompensująceych przesunięcie fazowe wynikające z rozciągłości przestrzenej. Pomimo wąskiego zakresu użytecznego parametrów, macierz neuronów FitzHugh-Nagumo z opóźnieniami w transmisji potrafi wzmonić sygnał periodyczny nawet w przypadku niewielkich fluktuacji różnic fazy.

Wyniki wskazują na dużą użyteczność i uniwersalność metody wzmocnienia SR w układach przestrzennie rozciągłych przy pomocy odpowiednich opóźnień w transmisji sygnału pomiędzy oddziałującymi elementami, stanowiącej pewną formę kontroli rezonansu stochastycznego.